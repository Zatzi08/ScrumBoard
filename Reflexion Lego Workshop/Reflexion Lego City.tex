\documentclass[a4paper, 11pt]{article}
\usepackage[top=3cm, bottom=3cm, left = 2cm, right = 2cm]{geometry} 
\geometry{a4paper} 
\usepackage[utf8]{inputenc}
\usepackage{textcomp}
\usepackage{graphicx}
\usepackage[ngerman]{babel} 
\usepackage{amsmath,amssymb}  
\usepackage{bm}
\usepackage{etoolbox}
\usepackage[pdftex,bookmarks,colorlinks,breaklinks,colorlinks=true,linkcolor=black,citecolor=gray,urlcolor=azure]{hyperref}  
%\hypersetup{linkcolor=black,citecolor=black,filecolor=black,urlcolor=black} % black links, for printed output
\usepackage{memhfixc} 
\usepackage{pdfsync}
\usepackage{booktabs}
\usepackage{fancyhdr}
\usepackage{enumitem}
\usepackage{enumerate}
\usepackage{pifont}
\usepackage{paralist}
\usepackage{tcolorbox}
\pagestyle{fancy}

\title{\Huge Reflexion Lego City}
\author{Freyschmidt, Henry Lewis & Hama, Zana Salih & Krasnovska, Paula & Krüger, Lucas & Prüger, Marvin Oliver & Seep, Tom-Malte & Zabel, Steven}
\date{\today}

\begin{document}
	\clearpage\maketitle
	\thispagestyle{empty}
	
	\vfill
	
	\begin{figure}[b!]

	\begin{compactitem}[\null]
		\item \textbf{Version:} 1
		%  Verantwortliche Mitglieder für Organisation, Anlegen der Dokumentenversion, Moderation, Abgabe
		\item \textbf{Verantwortliche Teammitglieder:}
		\begin{compactitem}
			\item Krüger, Lucas (Organisation)
			\item Zabel, Steven (Anlegen der Dokumentenversion)
			\item Krüger, Lucas (Abgabe) \\
		\end{compactitem}
		\item \textbf{Anwesende während der Meetings:}
		\begin{compactitem}
			\item  Freyschmidt, Henry Lewis
			\item  Hama, Zana Salih
			\item  Krasnovska, Paula
   		\item  Krüger, Lucas
     	\item  Prüger, Marvin Oliver
       	\item  Zabel, Steven\\
		\end{compactitem}
		\item \textbf{(Online-) Beiträge zum Inhalt durch:}
		\begin{compactitem}
			\item  Freyschmidt, Henry Lewis
            \item  Hama, Zana Salih
            \item Krasnovska, Paula
   		\item  Krüger, Lucas
     	\item  Prüger, Marvin Oliver
      	\item  Seep, Tom-Malte
       	\item  Zabel, Steven\\
		\end{compactitem}
		\item \textbf{Korrektur gelesen durch:}
		\begin{compactitem}
			\item Krüger, Lucas
            \item Krasnovska, Paula
      	\item Seep, Tom-Malte
			\item Hama, Zana Salih
		\end{compactitem}
	\end{compactitem}\unskip
\end{figure}

	\pagebreak
	\setcounter{page}{1}
	\tableofcontents
	
	\pagebreak
	
	\section{Reflexion zum Workshop 1}

\subsection{Herausforderungen} 
\begin{itemize}
    \item Ressourcen: Anschaffen und Verteilung der Bausteine; Knappheit mancher Bausteine
    \item Aufgaben: Priorität in Arbeitsaufwand abbilden; Abhängigkeit von Arbeitsaufwand und Zeit beachten; Koordinierung mit den Gruppen
    \item Stakeholdereinbindung: Klarheit über Anforderungen schaffen; Umsetzung der Anforderungen
    \item Kommunikation: Absprache zwischen Baugruppen und Orga.-Team; Absprache mit Stakeholder
    \item Anfang: Chaos der ersten Minuten kontrollieren
\end{itemize}

\subsection{Was gut lief}
\begin{itemize}
    \item Es entstand schnell ein Organisationsteam, welches die Aufgaben erfasst und verteilt hat.
    \item Es entstanden relativ schnell Teams, um an den Aufgaben zu arbeiten.
    \item Die meisten Anforderungen wurden erfüllt.
    \item Die Errichtung der Stadt fand modular statt.
    \item Es wurde eine Stadt gebaut, welche als solche identifizierbar war.
    \item Es gab Ansätze einer Gruppendynamik innerhalb der Bauenden. Man hatte zum Teil intuitiv eine Arbeitseinteilung vorgenommen nach dem Schema Ressourcenbeschaffende, Bauende oder auch hybride Varianten.
\end{itemize}

\subsection{Verbesserungspotential}

\begin{itemize}
    \item Das Größenverhältnis der Stadt und einzelner Bauteile war nicht festgelegt. Teilweise wurde es auf ein und derselben Platte nicht eingehalten.
    \item Es gab nur geringfügig Absprache mit dem Stakeholder.
    \item Die Bausteine waren unsortiert in einem Karton, wodurch sowohl das Finden der richtigen Steine, als auch das gemeinsame Suchen viel Zeit in Anspruch nahm und Staus beziehungsweise Platzprobleme verursachte.
    \item Es arbeiteten teilweise zu viele Leute an einer Platte, wodurch die Effizienz der Gruppe eingeschränkt war und man einander verhinderte.
    \item Die Baugruppen haben nach Bekommen der Aufgabe häufig ohne Rücksprache mit dem Stakeholder bis zum Schluss gearbeitet, wodurch beispielsweise ein bunter Plattenbau entstand, der nicht bunt sein sollte.
    \item Die Rollenverteilung und -umsetzung war nicht optimal, sodass beispielsweise das Projektmanagement keine Rücksprache mit den Bauenden hielt und keine Übersicht über das Gebaute hatte oder auch dass zu viele Bauende einer einzigen Aufgabe zugeteilt waren.
\end{itemize}
\pagebreak
	
\section{Änderungsvorschläge anhand agiler Praktiken}

\begin{itemize}
    \item (Struktur:) Änderungsvorschlag
    \begin{itemize}
        \item[\ding{227}] (Struktur:) Begründung
    \end{itemize}
    \item Der Stakeholder sollte, beispielsweise durch häufigere Absegnungen, mehr einbezogen werden. Hierfür sollte es zusätzlich eine Kundenberatung geben, welche im permanenten Kontakt zum Kunden steht und Wünsche des Kunden mit dem Organisationsteam bespricht.
    \begin{itemize}
        \item[\ding{227}] Einige Baugruppen waren vollends in den Bau vertieft und hörten nur von dem Stakeholder, wenn dieser aktiv auf sie zuging. Dadurch wurde viel Arbeit geschafft, ohne zu wissen, ob diese im Detail gewünscht war. Ein Beispiel dafür wäre unter anderem der bunte Plattenbau.
    \end{itemize}
    \item Die Gebäude können individuell auf kleinen Platten gebaut werden und im fertigen Zustand erst auf die Platte der Baugruppe gesetzt werden.
    \begin{itemize}
        \item[\ding{227}] Es arbeiteten teilweise zu viele Bauende an einer Platte, weshalb nicht alle zeitgleich aktiv sein konnten.
    \end{itemize}
    \item Es sollte eine Einschätzung der Aufgaben bezüglich des Arbeitsaufwands stattfinden, um die Bauenden besser einzuteilen.
    \begin{itemize}
        \item[\ding{227}] Es arbeiteten teilweise zu viele Bauende an einer Platte, weshalb nicht alle zeitgleich aktiv sein konnten.
    \end{itemize}
    \item Der stärkste Fokus beim Bauen sollte auf den höchsten Prioritäten liegen. Niedrigere Prioritäten sollten zunächst zwar mitgebaut, aber nicht detailreich ausgebaut werden.
    \begin{itemize}
        \item[\ding{227}] Der Bau von Gebäuden mit hoher Priorität wurde teilweise nicht abgeschlossen, weil zuvor zu viel Zeit in den Bau von Objekten niedrigerer Priorität wie Wege, Büsche oder Fahrradständer investiert wurde.
    \end{itemize}
    \item Es sollte Rollen im Baumanagement geben, welches die Teams koordiniert, mit dem Projektmanagement kommuniziert, eine grobe Bausteinverteilung und -umverteilung vornimmt und Bauvorgaben wie das Größenverhältnis abstimmt.
    \begin{itemize}
        \item[\ding{227}] Die Kommunikation zwischen den Baugruppen und dem Projektmanagement war schlecht, wodurch ungewollte Ergebnisse spät oder zur Bauphase gar nicht erkannt wurden. Zudem war der Karton mit den Bausteinen häufig für Baugruppen unerreichbar, da bereits zu viele Menschen dort Steine suchten.
    \end{itemize}
    \item Es sollte, bestenfalls parallel zur Aufgabenerfassung durch das Organisationsteam, eine Vorabsortierung und -verteilung der Bausteine geben.
    \begin{itemize}
        \item[\ding{227}] Dass alle Bausteine in einem Karton unsortiert gelagert waren, hat dafür gesorgt, dass bestimmte Steine lange gesucht wurden und ein ständig anhaltender Stau entstand, der aufgrund des begrenzten Platzes um den Karton herum nicht aufgelöst werden konnte. Das Suchen und Warten am Karton hat viel Arbeitszeit gekostet.
    \end{itemize}
\end{itemize}

	%TODO: Bitte % unter den nächsten beiden Befehlen entfernen falls Referenzen genutzt wurden
	%\bibliographystyle{abbrv}
	%\bibliography{references}  % need to put bibtex references in references.bib 
\end{document}