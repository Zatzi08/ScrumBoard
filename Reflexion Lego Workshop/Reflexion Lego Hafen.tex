\documentclass[a4paper, 11pt]{article}
\usepackage[top=3cm, bottom=3cm, left = 2cm, right = 2cm]{geometry} 
\geometry{a4paper} 
\usepackage[utf8]{inputenc}
\usepackage{textcomp}
\usepackage{graphicx}
\usepackage[ngerman]{babel} 
\usepackage{amsmath,amssymb}  
\usepackage{bm}
\usepackage{etoolbox}
\usepackage[pdftex,bookmarks,colorlinks,breaklinks,colorlinks=true,linkcolor=black,citecolor=gray,urlcolor=azure]{hyperref}  
%\hypersetup{linkcolor=black,citecolor=black,filecolor=black,urlcolor=black} % black links, for printed output
\usepackage{memhfixc} 
\usepackage{pdfsync}
\usepackage{booktabs}
\usepackage{fancyhdr}
\usepackage{enumitem}
\usepackage{enumerate}
\usepackage{array}
\usepackage{pifont}
\usepackage{paralist}
\usepackage{tcolorbox}	
\usepackage{color, colortbl}
\usepackage{array}
\pagestyle{fancy}

\definecolor{Gray}{gray}{0.9}

\title{\Huge Reflexion Lego II: Hafen}
\author{Freyschmidt, Henry Lewis & Hama, Zana Salih & Krasnovska, Paula & Krüger, Lucas & Prüger, Marvin Oliver & Seep, Tom-Malte & Zabel, Steven}
\date{\today}

\begin{document}
	\clearpage\maketitle
	\thispagestyle{empty}
	
	\vfill
	
	\begin{figure}[b!]

	\begin{compactitem}[\null]
		\item \textbf{Version:} 2
		%  Verantwortliche Mitglieder für Organisation, Anlegen der Dokumentenversion, Moderation, Abgabe
		\item \textbf{Verantwortliche Teammitglieder:}
		\begin{compactitem}
			\item Krüger, Lucas (Organisation)
			\item Hama, Zana Salih (Anlegen der Dokumentenversion)
                \item Krasnovska, Paula (Anlegen der Dokumentenversion)
			\item Krüger, Lucas (Abgabe) \\
		\end{compactitem}
		\item \textbf{Anwesende während der Meetings:}
		\begin{compactitem}
		      \item  Freyschmidt, Henry Lewis
		      \item  Hama, Zana Salih
		      \item  Krasnovska, Paula
   		    \item  Krüger, Lucas
     	      \item  Prüger, Marvin Oliver
       	    \item  Seep, Tom-Malte\\
		\end{compactitem}
		\item \textbf{(Online-) Beiträge zum Inhalt durch:}
		\begin{compactitem}
		      \item  Freyschmidt, Henry Lewis
                \item  Hama, Zana Salih
                \item  Krasnovska, Paula
   	        \item  Krüger, Lucas
     	      \item  Prüger, Marvin Oliver
      	    \item  Seep, Tom-Malte
       	    \item  Zabel, Steven\\
		\end{compactitem}
		\item \textbf{Korrektur gelesen durch:}
		\begin{compactitem}
                \item Hama, Zana Salih
                \item Zabel, Steven
                \item Krüger, Lucas
		\end{compactitem}
	\end{compactitem}\unskip
\end{figure}

	\pagebreak
	\setcounter{page}{1}
	\tableofcontents
	
        \pagebreak

\section{Reflexion zum Workshop 2}

\subsection{Herausforderungen}
\begin{itemize}
    \item Maßstab: Es wurde ein Maßstab anhand von Containern (ein 4 × 2 Brick) eingeführt. Dies führte allerdings zu Verwirrung und Ungenauigkeiten, wodurch beispielsweise der Wachturm im Vergleich zu den Kränen eine unrealistische Größe annahm.
    \item Rollen: Die Rollen wurden schnell in Scrum Master, Product Owner und Bauende eingeteilt, welche jedoch nicht korrekt oder kaum ausgeführt wurden. Somit war die Kommunikation zwischen der Projektplanung und den Bauenden mangelhaft.
    \item klare Aufgabenverteilung: Die Product Owner haben keine eindeutigen Aufgabenstellungen an die Bauenden weitergeleitet. Dadurch gab es viele Interpretationsmöglichkeiten für eine Aufgabe, weswegen oftmals inkorrekt gebaut wurde oder Anforderungen nicht erfüllt wurden.
    \item Stakeholder: Der Stakeholder war selten anwesend und ansprechbar, was einen Abgleich mit den Anforderungen erschwerte.
\end{itemize}

\subsection{Was gut lief}
\begin{itemize}
    \item Der Zugang zu den Ressourcen war besser, wodurch es weniger zu Engpässen während der Bausteinlieferung kam.
    \item Die Baugruppen waren kleiner, wodurch jedes Mitglied stets etwas zu tun hatte und die Aufgaben effizienter erledigt wurden.
    \item Es wurde ein Maßstab gesetzt, an welchem sich orientiert wurde.
    \item Es wurden alle Aufgaben in der gegebenen Zeit erledigt.
    \item Dem Stakeholder wurden Prototypen vorgestellt, welche bewertet und dadurch verändert werden konnten.
\end{itemize}

\subsection{Verbesserungspotential} 
\begin{itemize}
    \item Die Kommunikation zwischen den Product Ownern und dem Bauteam war mangelhaft.
    \item Die Integration der Platten verlief nicht einwandfrei und unter hohem Zeitdruck. Hätten die Teams vorher kommuniziert, wären am Ende weniger Probleme entstanden.
    \item Die Scrum Master und Product Owner hätten mehr Zeit gebraucht, um sich mit deren Aufgaben zu befassen, um diese korrekt anzuwenden.
    \item Obwohl uns Planning Poker bereitgestellt wurde, wurden die Aufgaben erneut nicht in Arbeitsaufwand und Prioritäten eingeteilt. Dies wurde auch ausschließlich von der Projektplanung besprochen, ohne die Teams selbst einzubeziehen. Am Ende wurden wieder die Aufgaben mit der größten Priorität bevorzugt, obwohl einige Aufgaben einen viel geringeren Aufwand hatten und somit schneller hätten erledigt werden können. Dies war nicht zeiteffizient.
\end{itemize}
\pagebreak



\section {Vergleich der beiden Workshops}
\begin{table}[h]
    \centering
    \renewcommand{\arraystretch}{1.3}
    \begin{tabular}{| p{5cm} | p{5cm} | p{5cm} |}
        \hline
        \rowcolor{Gray} 
        \multicolumn{1}{|c|}{\textbf{Verbesserung}} & \multicolumn{1}{c|}{\textbf{Keine Veränderung}} & \multicolumn{1}{c|}{\textbf{Verschlechterung}} \\
        \hline
        Ressourcenverteilung & Aufgabenzuweisung hat sehr lange gedauert & komplikationsreichere Integration der einzelnen Platten \\
        \hline
        kaum Engpässe bei gesuchten Bausteinen & Aufgaben nicht ausreichend in Teilaufgaben untergliedert & keine Visualisierung des Fortschritts \\
        \hline
        effizientere Arbeitsverteilung & Priorisierung der Aufgaben wurde nach höchster Priorität geordnet & \\
        \hline
        Bau effizienter & Aufgaben teilweise ungenau gestellt & \\
        \hline
        Maßstab und Rahmenbedingungen wurden besprochen & Bewältigung aller Aufgaben & \\
        \hline
        Stakeholder zur Bewertung von Prototypen miteinbezogen & & \\
        \hline
        Funktionalität der Baukörper eindeutig erkennbar & & \\
        \hline
    \end{tabular}
\end{table}
\pagebreak



\section{Eigener Prozessentwurf}
\subsection{Agile Praktiken}
\begin{itemize}
    \item Test-Driven Development
    \item Pair Programming, falls es für die gegebene Anforderung sinnvoll ist.
    \item Coding Standards
    \begin{itemize}
        \item schlüssige Kommentare
        \begin{itemize}
            \item verwendete Abkürzungen im Kommentar kenntlich machen
            \item Funktionalität einer Funktion kurz und knapp umschreiben
            \item Namenskürzel zur Kennzeichnung, wer an einem Code-Abschnitt arbeitete (zzgl. klare Kennzeichnung einer Revision)
        \end{itemize}
        \item Verwendung intuitiver Parameternamen ohne Raum für Ambiguität
        \item kompakte Code-Formatierung
        \begin{itemize}
            \item einzeilige, kurze if-else-Konstrukte auf einer Zeile
        \end{itemize}
        \item Funktionen so kompakt wie möglich halten, ansonsten: Unterteilung in Unterfunktionen
    \end{itemize}
    \item Anlegen eines Backlogs zur Förderung einer strukturierten Arbeitsweise
\end{itemize}
\subsection{Rollenverteilung}
\begin{itemize}
    \item Product Owner
    \begin{itemize}
        \item Zabel, Steven
    \end{itemize}
    \item Scrum Master
    \begin{itemize}
        \item Hama, Zana Salih
    \end{itemize}
    \item weiteres Team 
    \begin{itemize}
        \item Freyschmidt, Henry Lewis
        \item Krasnovska, Paula
        \item Krüger, Lucas
        \item Prüger, Marvin Oliver
        \item Seep, Tom-Malte
    \end{itemize}
\end{itemize}
\subsection{Zeiteinteilung}
\begin{itemize}
    \item Sprintplanung mit Planning Game/Poker
    \item wöchentliche Meetings im Stil von Daily Meetings
    \item Ende vom Sprint
    \begin{itemize}
        \item allumfassende Retrospektive/Review Meeting
        \item fertige Funktionen/vertikaler Prototyp
    \end{itemize}
\end{itemize}

\end{document}